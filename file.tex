const form = document.getElementById('formulario');
const userName = document.getElementById('userName');

form.addEventListener('submit', (event) =>{
  event.preventDefault();

  //CheckInputUserName();
  //para 'pegar' o que o usuário digita
  //deixei em comentário e a utilizei em checkForm  
  CheckForm();

})



userName.addEventListener('blur',() =>{
    CheckInputUserNameheckInputUserName();
})

function CheckInputUserName(){
const userNameValue = userName.value;
//agora conseguimos pegar o que o usuário digitou em 'userName'
if (userNameValue=== '') {
    errorInput(userName,'Preencha o espaço indicado.')
}else{
    const formItem = userName.parentElement;
    formItem.className = 'formulario-content'

    //acréscimo de minha autoria (porque uma parte do código não funcinava)
    alert('Publicação realizada com sucesso!');
}


function CheckForm(){
    CheckInputUserName();

    //agora realizaremos uma validação(para que não mostre o aviso de 
    //"Publicação realizada com sucesso!" sem que os dados estejam devidamente corretos)

    const formItens = form.querySelectorAll('formulario-content');
    //para pegar todos os elementos que tiverem a classe  formulario-content

    const isValid = [...formItens].every((item) =>{
        return item.className ==='formulario-content';
         //para que seja verificado, elemento por elemento(no caso temos apenas 1 elemento), se 
         // o input atende os requisitos exigidos

    });

    if(isValid){
        alert('Publicação realizada com sucesso!');
            }


}

function errorInput(input,message) {
    const formItem = input.parentElement;
    //para que o 'errorInput ' se direcione ao pai (que é 'formulario-content')
    const textMessage = formItem.querySelector('a');


    textMessage.innerText = message;

    formItem.className = 'formulario-content error';
}
}